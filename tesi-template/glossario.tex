
%**************************************************************
% Acronimi
%**************************************************************
\renewcommand{\acronymname}{Acronimi e abbreviazioni}

\newacronym[description={\glslink{apig}{Application Program Interface}}]
    {api}{API}{Application Program Interface}

\newacronym[description={\glslink{umlg}{Unified Modeling Language}}]
    {uml}{UML}{Unified Modeling Language}

%**************************************************************
% Glossario
%**************************************************************
%\renewcommand{\glossaryname}{Glossario}


\newglossaryentry{Junior}
{
	name=\glslink{Junior}{Junior},
	text=Junior,
	sort=Junior,
	description={in informatica con il termine \emph{Junior} si intende una figura professionale agli inizi di carriera senza significative esperienze regresse sul campo di interesse.}
}

\newglossaryentry{team}
{
	name=\glslink{team}{team},
	text=team,
	sort=team,
	description={in informatica con il termine \emph{team} si intende un gruppo di sviluppatori che collaborano in maniera organizzata per raggiungere un obbiettivo comune.}
}

\newglossaryentry{engine}
{
	name=\glslink{engine}{engine},
	text=racing,
	sort=racing,
	description={in informatica con il termine \emph{engine} si intende un sotware che funge da base per altri. Nello specifico dei videogame, l'engine è spesso scambiato per il solo motore grafico. Invece questo comprende praticamente tutte le funzioni di base che poi vengono specializzate per la creazione di specifici video game.}
}

\newglossaryentry{debug}
{
	name=\glslink{debug}{debug},
	text=debug,
	sort=debug,
	description={in informatica con il termine \emph{debug} si intende la pratica dell'analisi del comportamento del codice, soprattutto a run-time, allo scopo di individuare e risolvere bug (difetti).}
}

\newglossaryentry{refactoring}
{
	name=\glslink{refactoring}{refactoring},
	text=refactoring,
	sort=refactoring,
	description={in informatica con il termine \emph{refactoring} si indica la pratica di riorganizzazione del codice, allo scopo di migliorne le proprità.}
}

\newglossaryentry{TRC/TCR}
{
	name=\glslink{TRC/TCR}{TRC/TCR},
	text=TRC/TCR,
	sort=TRC/TCR,
	description={todo}
}

\newglossaryentry{Day One}
{
	name=\glslink{Day One}{Day One},
	text=Day One,
	sort=Day One,
	description={per \emph{day one} si intende il primo giorno in cui un gioco viene reso disponibile per l'acquisto.}
}

\newglossaryentry{hackerate}
{
	name=\glslink{hackerate}{hackerate},
	text=hackerate,
	sort=hackerate,
	description={nel constesto di questa tesi, il termine è stato usato per indicare la pratica di bucare il software delle console o dei videogiochi per poter giocare con copie non autorizzate.}
}

\newglossaryentry{ban}
{
	name=\glslink{ban}{ban},
	text=ban,
	sort=ban,
	description={in informatica con il termine \emph{ban} si intende l'esclusione, a tempo determinato o non, di un utente da un servizio di qualsivoglia genere.}
}

\newglossaryentry{PSN}
{
	name=\glslink{PSN}{PSN},
	text=PSN,
	sort=PSN,
	description={il Playstation Network (PSN) è il servizio offerto da Sony agli utenti delle proprie console. Esso permette di giocare online con altri giocatori e l'acquisto di contenuti digitali (film, giochi ecc.). }
}

\newglossaryentry{bug}
{
	name=\glslink{bug}{bug},
	text=bug,
	sort=bug,
	description={in informatica con il termine \emph{bug} si definisce un problema presente in un codice che porta questo a mostrare comportamenti indesiderati, ad esempio chrash improvvisi.}
}

\newglossaryentry{leaderboard}
{
	name=\glslink{leaderboard}{leaderboard},
	text=leaderboard,
	sort=leaderboard,
	description={nei videogiochi online, le \emph{leaderboard} sono delle classifiche sulle prestazioni dei giocatori. }
}

\newglossaryentry{middleware}
{
	name=\glslink{middleware}{middleware},
	text=middleware,
	sort=middleware,
	description={in informatica con il termine \emph{middleware} todo}
}

\newglossaryentry{Raknet}
{
	name=\glslink{Raknet}{Raknet},
	text=Raknet,
	sort=Raknet,
	description={todo}
}

\newglossaryentry{Microsoft Project}
{
	name=\glslink{Microsoft Project}{Microsoft Project},
	text=Microsoft Project,
	sort=Microsoft Project,
	description={todo}
}

\newglossaryentry{AAA}
{
	name=\glslink{AAA}{AAA},
	text=AAA,
	sort=AAA,
	description={nel mondo dei videogiochi i titoli \emph{AAA} (pronunciato \dq{tripla a}) definisco quei titoli di punta sviluppati dalle case più grandi con maggiori budget.}
}

\newglossaryentry{First Playable}
{
	name=\glslink{First Playable}{First Playable},
	text=First Playable,
	sort=First Playable,
	description={}
}

\newglossaryentry{Vertical Slice}
{
	name=\glslink{Vertical Slice}{Vertical Slice},
	text=Vertical Slice,
	sort=Vertical Slice,
	description={Nel mondo dei videogiochi con \emph{Vertical Slice} si intende una versione di un videogioco non ancora completata, ma presenta un assaggio completo di tutti gli strati che lo compongono. Gli esempi classico di vertical slice è ad esempio una demo in cui è permesso al giocatore di giocare un solo livello (su tanti che saranno) in maniera completa.}
}

\newglossaryentry{mesh}
{
	name=\glslink{mesh}{mesh},
	text=mesh,
	sort=mesh,
	description={nel settore della grafica 3D, il termine \emph{mesh} indica l'insieme dei poligoni (triangoli) che compongono un oggetto tridimensionale.}
}

\newglossaryentry{texture}
{
	name=\glslink{texture}{texture},
	text=texture,
	sort=texture,
	description={nella conoscenza comune la texture è l'immagine 2D che viene applicata alle mesh per farle sembrare veri oggetti. Nell'ambito invece della programmazione il termine assume invece un contesto più generale, significando una matriche di punti (considerabile come un'immagine) in cui però viene utilizzata per contenere molti altre cose. todo rivedere}
}

\newglossaryentry{XML}
{
	name=\glslink{XML}{XML},
	text=XML,
	sort=XML,
	description={todo}
}

\newglossaryentry{LUA}
{
	name=\glslink{LUA}{LUA},
	text=LUA,
	sort=LUA,
	description={in informatica con il termine \emph{Application Programming Interface API} (ing. interfaccia di programmazione di un'applicazione) si indica ogni insieme di procedure disponibili al leaderboard, di solito raggruppate a formare un set di strumenti specifici per l'espletamento di un determinato compito all'interno di un certo programma. La finalità è ottenere un'astrazione, di solito tra l'hardware e il programmatore o tra software a basso e quello ad alto livello semplificando così il lavoro di programmazione}
}

\newglossaryentry{blend tree}
{
	name=\glslink{blend tree}{blend tree},
	text=blend tree,
	sort=blend tree,
	description={in informatica con il termine \emph{Application Programming Interface API} (ing. interfaccia di programmazione di un'applicazione) si indica ogni insieme di procedure disponibili al leaderboard, di solito raggruppate a formare un set di strumenti specifici per l'espletamento di un determinato compito all'interno di un certo programma. La finalità è ottenere un'astrazione, di solito tra l'hardware e il programmatore o tra software a basso e quello ad alto livello semplificando così il lavoro di programmazione}
}

\newglossaryentry{Perforce}
{
	name=\glslink{Perforce}{Perforce},
	text=Perforce,
	sort=Perforce,
	description={}
}

\newglossaryentry{Alienware}
{
	name=\glslink{Alienware}{Alienware},
	text=Alienware,
	sort=Alienware,
	description={}
}

\newglossaryentry{delta}
{
	name=\glslink{delta}{delta},
	text=delta,
	sort=delta,
	description={}
}

\newglossaryentry{serializzazione}
{
	name=\glslink{serializzazione}{serializzazione},
	text=serializzazione,
	sort=serializzazione,
	description={}
}

\newglossaryentry{sistema di integrazione continua}
{
	name=\glslink{sistema di integrazione continua}{sistema di integrazione continua},
	text=sistema di integrazione continua,
	sort=sistema di integrazione continua,
	description={}
}

\newglossaryentry{build}
{
	name=\glslink{build}{build},
	text=build,
	sort=build,
	description={}
}