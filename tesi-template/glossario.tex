
%**************************************************************
% Acronimi
%**************************************************************
\renewcommand{\acronymname}{Acronimi e abbreviazioni}

\newacronym[description={\glslink{apig}{Application Program Interface}}]
    {api}{API}{Application Program Interface}

\newacronym[description={\glslink{umlg}{Unified Modeling Language}}]
    {uml}{UML}{Unified Modeling Language}

%**************************************************************
% Glossario
%**************************************************************
%\renewcommand{\glossaryname}{Glossario}

\newglossaryentry{racing}
{
    name=\glslink{racing}{racing},
    text=racing,
    sort=racing,
    description={in informatica con il termine \emph{Application Programming Interface API} (ing. interfaccia di programmazione di un'applicazione) si indica ogni insieme di procedure disponibili al programmatore, di solito raggruppate a formare un set di strumenti specifici per l'espletamento di un determinato compito all'interno di un certo programma. La finalità è ottenere un'astrazione, di solito tra l'hardware e il programmatore o tra software a basso e quello ad alto livello semplificando così il lavoro di programmazione}
}

\newglossaryentry{Junior}
{
	name=\glslink{Junior}{Junior},
	text=Junior,
	sort=Junior,
	description={in informatica con il termine \emph{Application Programming Interface API} (ing. interfaccia di programmazione di un'applicazione) si indica ogni insieme di procedure disponibili al programmatore, di solito raggruppate a formare un set di strumenti specifici per l'espletamento di un determinato compito all'interno di un certo programma. La finalità è ottenere un'astrazione, di solito tra l'hardware e il programmatore o tra software a basso e quello ad alto livello semplificando così il lavoro di programmazione}
}

\newglossaryentry{team}
{
	name=\glslink{team}{team},
	text=team,
	sort=team,
	description={in informatica con il termine \emph{Application Programming Interface API} (ing. interfaccia di programmazione di un'applicazione) si indica ogni insieme di procedure disponibili al programmatore, di solito raggruppate a formare un set di strumenti specifici per l'espletamento di un determinato compito all'interno di un certo programma. La finalità è ottenere un'astrazione, di solito tra l'hardware e il programmatore o tra software a basso e quello ad alto livello semplificando così il lavoro di programmazione}
}

\newglossaryentry{engine}
{
	name=\glslink{engine}{engine},
	text=racing,
	sort=racing,
	description={in informatica con il termine \emph{Application Programming Interface API} (ing. interfaccia di programmazione di un'applicazione) si indica ogni insieme di procedure disponibili al programmatore, di solito raggruppate a formare un set di strumenti specifici per l'espletamento di un determinato compito all'interno di un certo programma. La finalità è ottenere un'astrazione, di solito tra l'hardware e il programmatore o tra software a basso e quello ad alto livello semplificando così il lavoro di programmazione}
}

\newglossaryentry{debug}
{
	name=\glslink{debug}{debug},
	text=debug,
	sort=debug,
	description={in informatica con il termine \emph{Application Programming Interface API} (ing. interfaccia di programmazione di un'applicazione) si indica ogni insieme di procedure disponibili al programmatore, di solito raggruppate a formare un set di strumenti specifici per l'espletamento di un determinato compito all'interno di un certo programma. La finalità è ottenere un'astrazione, di solito tra l'hardware e il programmatore o tra software a basso e quello ad alto livello semplificando così il lavoro di programmazione}
}

\newglossaryentry{refactoring}
{
	name=\glslink{refactoring}{refactoring},
	text=refactoring,
	sort=refactoring,
	description={in informatica con il termine \emph{Application Programming Interface API} (ing. interfaccia di programmazione di un'applicazione) si indica ogni insieme di procedure disponibili al programmatore, di solito raggruppate a formare un set di strumenti specifici per l'espletamento di un determinato compito all'interno di un certo programma. La finalità è ottenere un'astrazione, di solito tra l'hardware e il programmatore o tra software a basso e quello ad alto livello semplificando così il lavoro di programmazione}
}