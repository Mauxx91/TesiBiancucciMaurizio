% !TEX encoding = UTF-8
% !TEX TS-program = pdflatex
% !TEX root = ../tesi.tex
% !TEX spellcheck = it-IT

%**************************************************************
\chapter{Introduzione}
\label{cap:introduzione}
%**************************************************************

%**************************************************************
\section{Descrizione dell'azienda}

Milestone S.r.l. (MI) nasce a Milano nel 1996, e ancora oggi rappresenta la più grande realtà italiana impegnata nello sviluppo di videogiochi per console e PC. L'azienda è riconosciuta a livello mondiale come uno dei migliori team di sviluppo nel settore \gls{racing}, sia asfalto che off-road, sia motociclismo che automobilismo\cite{site:milestone}.\\

Fra i suoi titoli più importanti si vuol ricordare \dq{SCAR - Squadra Corse Alfa Romeo}, \dq{MXGP}, \dq{WRC 4 Word Rally Championship}, \dq{la serie Superbike}, \dq{la serie MotoGP}. Tutti sviluppati sotto licenze ufficiali.\\

Al momento la casa è impegnata nella sfida dello sviluppo sulle nuove console, affrontando la sfida delle nuove tecnologie, cercando al contempo di mantenere lo stesso ritmo produttivo. 

%**************************************************************
\section{Motivazioni all'attivazione dello stage}

Lo stage nasce dall'esigenza dell'azienda, in fase di forte espansione, di aumentare il proprio personale, investendo quindi in formazione di figure \gls{Junior} da inserire successivamente in pianta stabile nell'organico dell'azienda.\\

Vista la sempre crescente realisticità grafica dei videogame, l'azienda cerca figure da inserire all'interno del \gls{team} di programmazione 3D.

%**************************************************************
\section{Obiettivi dello stage}

L'obiettivo dello stage è quello di inserire lo studente all'interno del team di programmazione 3D e fargli assaggiare le sfide con cui il team si confronta quotidianamente.\\

Lo stage prevede quindi di introdurre lo studente ai problemi affrontati chiedendogli di: analizzare, progettare e implementare dei tool di supporto allo scopo di velocizzare il lavoro dell'intero team.\\

La seconda parte dello stage prevede lo studio ad alto livello del funzionamento di un \gls{engine} di gioco, in particolare i flussi presenti nel settore di pertinenza del team. Seguito dalla progettazione ed implementazione di alcune funzionalità grafiche di \gls{debug}. È prevista la possibilità, in fase di progettazione di attuare del \gls{refactoring} sul codice di debug grafico già presente, allo scopo di una maggiore manutenibiltà ed estendibilità futura.

%**************************************************************
\section{Scopo del documento}

Il presente documento ha lo scopo di presentare gli esisti delle attività sostenute dallo studente durante il periodo di stage sostenuto presso Milestone S.r.l.\\

\begin{itemize}
	\item Nel capitolo \hyperref[cap:milestone]{2} verranno presentati i flussi di lavoro aziendali che portano un videogame da essere una semplice idea a diventare un prodotto maturo e finito pronto alla vendita. Particolare attenzione verrà dedicata alle responsabilità del team di programmazione 3D.
	
	\item Nel capito \hyperref[cap:engine-di-gioco]{3} verrà discussa la struttura ad alto livello di un engine di gioco.
	
	\item Nel capitolo \hyperref[cap:multiplatform-file-analyzer]{4} si presenterà il tool Multiplatform File Analyzer, analizzandolo approfonditamente tutti gli aspetti.
	
	\item Nel capitolo \hyperref[cap:xml-editor]{5} verrà presentato il tool XML Editor, del quale sarà fornita una profonda analisi a partire dall'analisi dei requisiti alla progettazione.
\end{itemize}

%**************************************************************
\section{Organizzazione del testo}

Per tutte le parole e le sequenze di parole in italico seguite dal carattere "g" a pedice, è presente un piccolo approfondimento nel Glossario in Appendice.

%**************************************************************
\section{Nota sul codice prodotto}

Il codice dei tool sviluppati, in accordo con l'azienda sono stati pubblicati sulla piattaforma \hyperref{https://github.com/}{}{}{GitHub}\footnote{Si invita il lettore a guardare nei capitoli \hyperref[cap:multiplatform-file-analyzer]{4} e \hyperref[cap:xml-editor]{5} sotto la sezione codice, verificare se e la sezione corretta, gli indirizzi dove reperire il codice) sotto licenza GNU GPL v3.}.
(todo mettere lo stesso link presente durante la spiegazione della licenza nei capitoli) (todo glossarizzare)\\

Riguardo il codice sviluppato all'interno dell'engine di gioco, è stato inserito il minimo necessario per permettere al lettore di comprendere e al contempo non rivelare informazioni sensibili dell'azienda. Tutto il codice di gioco presente all'interno di questo documento è materiale protetto da copyright appartenente a Milestone S.r.l. ed è pubblicato sotto autorizzazione. Ogni riproduzione è severamente vietata, ogni richiesta di ulteriore documentazione va recapitata direttamente a Milestone S.r.l.