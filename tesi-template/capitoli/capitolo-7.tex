% !TEX encoding = UTF-8
% !TEX TS-program = pdflatex
% !TEX root = ../tesi.tex
% !TEX spellcheck = it-IT

%**************************************************************
\chapter{Conclusioni}
\label{cap:conclusioni}
%**************************************************************

In questo capitolo verranno presentate le riflessioni scaturite dall'analisi effettuata a posteriori dell'esperienza di stage. Verranno analizzati gli obbiettivi raggiunti e l'esperienza acquisita durante il lavoro, concludendo con una valutazione di come la preparazione accademica si interfacci con il mondo del lavoro.

%**************************************************************
\section{Consuntivo finale}

%**************************************************************
\section{Raggiungimento degli obiettivi}

Lo studente ha raggiunto tutti gli obbiettivi pianificati ad inizio stage, con soddisfazione dell'azienda e del tutor interno. Questa infatti ha proposto altri sei mesi di collaborazione allo studente per continuare il percorso iniziato e continuare la formazione on-the-job.\\

I tool sviluppati durante la prima parte dello stage sono infatti stati apprezzati e verranno presto inseriti nel modus operandi aziendale. In essi lo studente ha raggiunto tutti i requisiti posti in fase di analisi degli stessi, seppur sforando di 4 giorni lavorativi sul tempo pianificato per la conclusione definitiva di XML Editor. Tempo recuperato poi durante lo sviluppo della seconda parte.\\

%**************************************************************
\section{Conoscenze acquisite e valutazione dei requisiti }

Lo studente ha approfondito la sua conoscenza delle librerie Qt, soprattutto ha potuto maturare esperienza nella progettazione di un'architettura che si adattasse bene al Framework in questione e che risultasse facilmente comprensibile ed estensibile. Egli ha potuto inoltre mettere in pratica le conoscenze di usabilità nel progettare la user interface dei tool, acquisite nei corsi di Tecnologie Web.



%**************************************************************
\section{Valutazione personale}
