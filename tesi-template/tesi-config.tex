%**************************************************************
% file contenente le impostazioni della tesi
%**************************************************************

%**************************************************************
% Impostazioni di header
%**************************************************************
\pagestyle{fancy}
\fancyhead[RE,LO]{\miniaturalogo}
\setlength{\headsep}{1.25cm}
\newcommand{\miniaturalogo}{\includegraphics[height = 1.25cm]{immagini/logo-unipd}}

%**************************************************************
% Impostazioni di impaginazione
% see: http://wwwcdf.pd.infn.it/AppuntiLinux/a2547.htm
%**************************************************************

\setlength{\parindent}{14pt}   % larghezza rientro della prima riga
\setlength{\parskip}{0pt}   % distanza tra i paragrafi

%**************************************************************
% Impostazioni di biblatex
%**************************************************************
\bibliography{bibliografia} % database di biblatex 

\defbibheading{bibliography}
{
    \cleardoublepage
    \phantomsection 
    \addcontentsline{toc}{chapter}{\bibname}
    \chapter*{\bibname\markboth{\bibname}{\bibname}}
}

\renewcommand*{\labelalphaothers}{\textsuperscript{+}}

\setlength\bibitemsep{1.5\itemsep} % spazio tra entry

\DeclareBibliographyCategory{opere}
\DeclareBibliographyCategory{web}

\defbibheading{opere}{\section*{Riferimenti bibliografici}}
\defbibheading{web}{\section*{Siti Web consultati}}


%**************************************************************
% Impostazioni di caption
%**************************************************************
\captionsetup{
    tableposition=top,
    figureposition=bottom,
    font=small,
    format=hang,
    labelfont=bf
}

%**************************************************************
% Impostazioni di glossaries
%**************************************************************

%**************************************************************
% Acronimi
%**************************************************************
\renewcommand{\acronymname}{Acronimi e abbreviazioni}

\newacronym[description={\glslink{apig}{Application Program Interface}}]
    {api}{API}{Application Program Interface}

\newacronym[description={\glslink{umlg}{Unified Modeling Language}}]
    {uml}{UML}{Unified Modeling Language}

%**************************************************************
% Glossario
%**************************************************************
%\renewcommand{\glossaryname}{Glossario}

\newglossaryentry{racing}
{
    name=\glslink{racing}{racing},
    text=racing,
    sort=racing,
    description={in informatica con il termine \emph{Application Programming Interface API} (ing. interfaccia di programmazione di un'applicazione) si indica ogni insieme di procedure disponibili al programmatore, di solito raggruppate a formare un set di strumenti specifici per l'espletamento di un determinato compito all'interno di un certo programma. La finalità è ottenere un'astrazione, di solito tra l'hardware e il programmatore o tra software a basso e quello ad alto livello semplificando così il lavoro di programmazione}
}

\newglossaryentry{Junior}
{
	name=\glslink{Junior}{Junior},
	text=Junior,
	sort=Junior,
	description={in informatica con il termine \emph{Application Programming Interface API} (ing. interfaccia di programmazione di un'applicazione) si indica ogni insieme di procedure disponibili al programmatore, di solito raggruppate a formare un set di strumenti specifici per l'espletamento di un determinato compito all'interno di un certo programma. La finalità è ottenere un'astrazione, di solito tra l'hardware e il programmatore o tra software a basso e quello ad alto livello semplificando così il lavoro di programmazione}
}

\newglossaryentry{team}
{
	name=\glslink{team}{team},
	text=team,
	sort=team,
	description={in informatica con il termine \emph{Application Programming Interface API} (ing. interfaccia di programmazione di un'applicazione) si indica ogni insieme di procedure disponibili al programmatore, di solito raggruppate a formare un set di strumenti specifici per l'espletamento di un determinato compito all'interno di un certo programma. La finalità è ottenere un'astrazione, di solito tra l'hardware e il programmatore o tra software a basso e quello ad alto livello semplificando così il lavoro di programmazione}
}

\newglossaryentry{engine}
{
	name=\glslink{engine}{engine},
	text=racing,
	sort=racing,
	description={in informatica con il termine \emph{Application Programming Interface API} (ing. interfaccia di programmazione di un'applicazione) si indica ogni insieme di procedure disponibili al programmatore, di solito raggruppate a formare un set di strumenti specifici per l'espletamento di un determinato compito all'interno di un certo programma. La finalità è ottenere un'astrazione, di solito tra l'hardware e il programmatore o tra software a basso e quello ad alto livello semplificando così il lavoro di programmazione}
}

\newglossaryentry{debug}
{
	name=\glslink{debug}{debug},
	text=debug,
	sort=debug,
	description={in informatica con il termine \emph{Application Programming Interface API} (ing. interfaccia di programmazione di un'applicazione) si indica ogni insieme di procedure disponibili al programmatore, di solito raggruppate a formare un set di strumenti specifici per l'espletamento di un determinato compito all'interno di un certo programma. La finalità è ottenere un'astrazione, di solito tra l'hardware e il programmatore o tra software a basso e quello ad alto livello semplificando così il lavoro di programmazione}
}

\newglossaryentry{refactoring}
{
	name=\glslink{refactoring}{refactoring},
	text=refactoring,
	sort=refactoring,
	description={in informatica con il termine \emph{Application Programming Interface API} (ing. interfaccia di programmazione di un'applicazione) si indica ogni insieme di procedure disponibili al programmatore, di solito raggruppate a formare un set di strumenti specifici per l'espletamento di un determinato compito all'interno di un certo programma. La finalità è ottenere un'astrazione, di solito tra l'hardware e il programmatore o tra software a basso e quello ad alto livello semplificando così il lavoro di programmazione}
} % database di termini
\newcommand{\gloss}[1]{\textit{\glstext{#1}}\ped{|\textit{g}|}}
\newcommand{\Csharp}{C{\#}}
\makeglossaries


%**************************************************************
% Impostazioni di graphicx
%**************************************************************
\graphicspath{{immagini/}} % cartella dove sono riposte le immagini


%**************************************************************
% Impostazioni di hyperref
%**************************************************************
\hypersetup{
    %hyperfootnotes=false,
    %pdfpagelabels,
    %draft,	% = elimina tutti i link (utile per stampe in bianco e nero)
    colorlinks=true,
    linktocpage=true,
    pdfstartpage=1,
    pdfstartview=FitV,
    % decommenta la riga seguente per avere link in nero (per esempio per la stampa in bianco e nero)
    %colorlinks=false, linktocpage=false, pdfborder={0 0 0}, pdfstartpage=1, pdfstartview=FitV,
    breaklinks=true,
    pdfpagemode=UseNone,
    pageanchor=true,
    pdfpagemode=UseOutlines,
    plainpages=false,
    bookmarksnumbered,
    bookmarksopen=true,
    bookmarksopenlevel=1,
    hypertexnames=true,
    pdfhighlight=/O,
    %nesting=true,
    %frenchlinks,
    urlcolor=webbrown,
    linkcolor=black,
    citecolor=webgreen,
    %pagecolor=RoyalBlue,
    %urlcolor=Black, linkcolor=Black, citecolor=Black, %pagecolor=Black,
    pdftitle={\myTitle},
    pdfauthor={\textcopyright\ \myName, \myUni, \myFaculty},
    pdfsubject={},
    pdfkeywords={},
    pdfcreator={pdfLaTeX},
    pdfproducer={LaTeX}
}

%**************************************************************
% Impostazioni di itemize
%**************************************************************
\renewcommand{\labelitemi}{$\bullet$}

%\renewcommand{\labelitemi}{$\bullet$}
%\renewcommand{\labelitemii}{$\cdot$}
%\renewcommand{\labelitemiii}{$\diamond$}
%\renewcommand{\labelitemiv}{$\ast$}

%**************************************************************
% Impostazioni di xcolor
%**************************************************************
\definecolor{webgreen}{rgb}{0,.5,0}
\definecolor{webbrown}{rgb}{.6,0,0}

%Visual studio 12 dark
%\definecolor{KeywordColor}{RGB}{86,156,214}
%\definecolor{TypeColor}{RGB}{78,201,176}
%\definecolor{MacroColor}{RGB}{189,99,197}
%\definecolor{StringColor}{RGB}{214,157,133}
%\definecolor{CommentColor}{RGB}{96,139,78}
%\definecolor{StaticColor}{RGB}{189,99,197}


%Visual studio 12 light
\definecolor{KeywordColor}{RGB}{0,0,255}
\definecolor{TypeColor}{RGB}{43,145,175}
\definecolor{MacroColor}{RGB}{111,0,138}
\definecolor{StringColor}{RGB}{163,21,21}
\definecolor{CommentColor}{RGB}{0,128,0}
\definecolor{StaticColor}{RGB}{189,99,197}
\definecolor{NumberColor}{RGB}{255,140,0}

%**************************************************************
% Impostazioni di listings
%**************************************************************
\lstset{
    language=[Visual]C++,
    keywordstyle=\ttfamily\color{KeywordColor}, %\bfseries,
    basicstyle=\ttfamily\small,
    commentstyle=\ttfamily\color{Green},
    numbers=left, %left, none%
    numberstyle=\ttfamily\small\color{TypeColor}, %\tiny
    stringstyle=\ttfamily\color{StringColor},
    stepnumber=1,
    numbersep=8pt,
    showstringspaces=false,
    breaklines=true,
    frameround=ftff,
    frame=single,
    tabsize = 4,
    literate={\ \ }{{\ }}1
} 

\lstdefinestyle{maurizio-code}{
	language =[Visual]C++,
	commentstyle = \ttfamily\color{CommentColor},
	keywordstyle={[2]\ttfamily\color{MacroColor}},
	keywordstyle={[3]\ttfamily\color{TypeColor}},
	keywordstyle={[4]\ttfamily\color{StaticColor}},
	morekeywords={[2]
		META\_BEGIN\_CLASS,
		META\_BEGIN\_PROPERTIES,
		META\_NAMED\_PROPERTY,
		META\_END\_PROPERTIES,
		META\_BEGIN\_ACTIONS,
		META\_ACTION,
		META\_END\_ACTIONS,
		META\_BEGIN\_EVENTS,
		META\_EVENT,
		META\_END\_EVENTS,
		META\_END\_CLASS,
		META\_PROPERTY,
		START\_META,
		END\_META,
		START\_ACTIONS,
		ACTION,
		END\_ACTIONS,
		ENABLE\_METADATA,
		CLSID\_MENUWORLD,
		CLSID\_GAMEWORLD,
		eWORLD\_ENABLE\_RENDER,
		eWORLD\_ENABLE\_ALL,
		GTRUE,
		GFALSE,
		CLSID\_ANIMATEDMODEL3D,
		GEM\_ASSERT,
		NULL,
		CLSID\_SKINCONTROLLER,
		g\_uiCLSID\_BUFFER\_CONTROLLER,
		SKELETON\_DEBUG\_DRAWER\_MANAGER\_H,
		GEM\_NEW,
		SKELETON\_DEBUG\_DRAWER\_H,
		RESOURCES\_MANAGER\_DEBUG,
		DEBUG\_AXIS\_DRAWER\_H,
		META\_NAMED\_GETTER
	},
	morekeywords={[3]
		void,
		int,
		float,
		TypedEvent,
		GString,
		CCamera,
		GVector3,
		GUInt32,
		GUInt8,
		GUInt64,
		GReal,
		GUInt,
		GInt,
		GColor,
		GBool,
		size\_t,
		vector,
		LodSelectionModule,
		CWorld,
		GraphicComponentInfo,
		FixedString256,
		FixedString64,
		GVector2,
		DebugTextDrawer,
		GemScreenLogger,
		DebugTextDrawer,
		DebugTextDrawer3DPoint,
		Config,
		DebugAxisDrawer,
		CEngine,
		GMatrix,
		CWorldManager,
		CML\_DrawVector,
		ResourcesManagerDebug,
		ObjectInfo,
		ObjectInfoContainer,
		ManagerInfo,
		ResourceInfo,
		ResourcesManager,
		CMutex,
		DebugData,
		CMutexAutoLock,
		ResourcesGroup,
		TManager,
		CObject3d,
		CObjects3dManager,
		uintptr\_t,
		uint32\_t,
		uint\_t,
		SkeletonDebugDrawer,
		CSkeleton,
		BufferController,
		RiderAnimationsBehaviour,
		Children,
		const\_iterator,
		SkeletonDebugDrawerManager,
		DebugWorldElement,
		ObjectDrawedInfo,
		CAnimatedModel3d,
		DrawParams,
		CML\_MetadataRegistry,
		SkinController
	},
	morekeywords={[4]
		Create,
		GetInstance,
		DrawVector,
		DrawSkeleton,
		DrawEMFXSkeleton,
		DrawPoint,
		GetJointLocalMatrix,
		GetJointColor,
		s\_instance,
		AddEntry,
		iterator
	}
}

%**************************************************************
% Altro
%**************************************************************

\newcommand{\omissis}{[\dots\negthinspace]} % produce [...]

% eccezioni all'algoritmo di sillabazione
\hyphenation
{
    ma-cro-istru-zio-ne
    gi-ral-din
    Xml-Relation-Checker-Core-Base
}

\expandafter\def\expandafter\UrlBreaks\expandafter{\UrlBreaks%  save the current one
	\do\a\do\b\do\c\do\d\do\e\do\f\do\g\do\h\do\i\do\j%
	\do\k\do\l\do\m\do\n\do\o\do\p\do\q\do\r\do\s\do\t%
	\do\u\do\v\do\w\do\x\do\y\do\z\do\A\do\B\do\C\do\D%
	\do\E\do\F\do\G\do\H\do\I\do\J\do\K\do\L\do\M\do\N%
	\do\O\do\P\do\Q\do\R\do\S\do\T\do\U\do\V\do\W\do\X%
	\do\Y\do\Z\do\0\do\1\do\2\do\3\do\4\do\5\do\6\do\7%
	\do\8\do\9}

\newcommand{\sectionname}{sezione}
\addto\captionsitalian{\renewcommand{\figurename}{figura}
                       \renewcommand{\tablename}{tabella}}

\newcommand{\intro}[1]{\emph{\textsf{#1}}}

%**************************************************************
% Environment per ``rischi''
%**************************************************************
\newcounter{riskcounter}                % define a counter
\setcounter{riskcounter}{0}             % set the counter to some initial value

%%%% Parameters
% #1: Title
\newenvironment{risk}[1]{
    \refstepcounter{riskcounter}        % increment counter
    \par \noindent                      % start new paragraph
    \textbf{\arabic{riskcounter}. #1}   % display the title before the 
                                        % content of the environment is displayed 
}{
    \par\medskip
}

\newcommand{\riskname}{Rischio}

\newcommand{\riskdescription}[1]{\textbf{\\Descrizione:} #1.}

\newcommand{\risksolution}[1]{\textbf{\\Soluzione:} #1.}

%**************************************************************
% Environment per ``use case''
%**************************************************************
\newcounter{usecasecounter}             % define a counter
\setcounter{usecasecounter}{0}          % set the counter to some initial value

%%%% Parameters
% #1: ID
% #2: Nome
\newenvironment{usecase}[2]{
    \renewcommand{\theusecasecounter}{\usecasename #1}  % this is where the display of 
                                                        % the counter is overwritten/modified
    \refstepcounter{usecasecounter}             % increment counter
    \vspace{10pt}
    \par \noindent                              % start new paragraph
    {\large \textbf{\usecasename #1: #2}}       % display the title before the 
                                                % content of the environment is displayed 
    \medskip
}{
    \medskip
}

\newcommand{\usecasename}{UC}

\newcommand{\usecaseactors}[1]{\textbf{\\Attori Principali:} #1. \vspace{4pt}}
\newcommand{\usecasepre}[1]{\textbf{\\Precondizioni:} #1. \vspace{4pt}}
\newcommand{\usecasedesc}[1]{\textbf{\\Descrizione:} #1. \vspace{4pt}}
\newcommand{\usecasepost}[1]{\textbf{\\Postcondizioni:} #1. \vspace{4pt}}
\newcommand{\usecasealt}[1]{\textbf{\\Scenario Alternativo:} #1. \vspace{4pt}}

%**************************************************************
% Environment per ``namespace description''
%**************************************************************

\newenvironment{namespacedesc}{
    \vspace{10pt}
    \par \noindent                              % start new paragraph
    \begin{description} 
}{
    \end{description}
    \medskip
}

\newcommand{\classdesc}[2]{\item[\textbf{#1:}] #2}

%**************************************************************
% Paragrafo
%**************************************************************

\newcommand{\paragrafo} {
	\hspace*{1cm}
}

%**************************************************************
% Apici e virgolette
%**************************************************************

\newcommand{\dq}[1] % testo
{\textquotedblleft#1\textquotedblright}

\newcommand{\sq}[1] % testo
{\textquoteleft#1\textquoteright}

%**************************************************************
% Tabella
%**************************************************************
\setlength{\tabcolsep}{6pt}
\renewcommand{\arraystretch}{1.75}

%**************************************************************
% Colore
%**************************************************************
\definecolor{headcolor}{RGB}{56,169,164}
\definecolor{subheadercolor}{RGB}{140,213,219}
\definecolor{row1}{RGB}{225,244,243}
\definecolor{row2}{RGB}{184,231,228}